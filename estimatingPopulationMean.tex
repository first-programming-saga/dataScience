\documentclass[a4j,12pt]{ltjarticle}
\usepackage{fancybox}
\usepackage{float}
\usepackage{graphicx}
\usepackage{amsmath,amssymb}
\usepackage{algorithm}
\usepackage{algpseudocode}
\usepackage{url}
\title{母集団平均の推定}
\author{}
\date{2020/10}
\begin{document}
\maketitle
\section{標本平均の分布}
平均$\mu$、標準偏差$\sigma$の母集団から、$n$個のサンプル$\left\lbrace x_i\right\rbrace$を無作為抽出する。
標本平均と標本標準偏差は
\begin{align}
    \overline{x}&=\frac{1}{n}\sum_{i=0}^{n-1}x_i\\
    s^2&=\frac{1}{n-1}\left(x_i-\overline{x}\right)^2
\end{align}
である。

標本平均は、平均$\mu$、標準偏差$\sigma/\sqrt{n}$の正規分布に従う。
これは、母集団の分布に依らない(中心極限定理)。

実際に、$[0,1)$の一様分布に対して、標本サイズ100で標本平均の分布を求める。
母集団の平均は$mu=1/2$、分散は$\sigma^2=1/12$である。
図~\ref{normal}に示すように、平均$\mu$、標準偏差$\sigma/\sqrt{n}$の正規分布となっている。
\begin{figure}[ht]
    \centering
    \includegraphics[width=80mm]{sampleMean.pdf}
    \caption{$[0,1)$の一様分布に対する標本平均の分布。標本サイズは100}
    \label{normal}
\end{figure}

\section{母集団平均の推定}
一つの標本から、母集団の平均を推定することを考える。
一つの標本であるため、標本平均と標本標準偏差を使うしかない。
標本平均が、平均$\mu$、標準偏差$s/\sqrt{n}$の正規分布に従うとする。

信頼係数を$c$とする。例えば、$c=.95$の場合、標本平均は、
\begin{equation}
-f\frac{s}{\sqrt{n}}<\overline{x}-\mu<f\frac{s}{\sqrt{n}}
\end{equation}
の範囲に入る。ここで、$f\simeq1.96$である。
これを変形することで、標本平均から信頼係数に依存した区間を推定することができる。
\begin{equation}
    -\frac{fs}{\sqrt{n}}+\overline{x}<\mu<\frac{fs}{\sqrt{n}}+\overline{x}
\end{equation}

\begin{figure}[ht]
    \centering
    \includegraphics[width=80mm]{populationMean.pdf}
    \caption{$[0,1)$の一様分布に対する標本平均から、母集団の平均の範囲を$95\%$で推計}
    \label{populationMean}
\end{figure}

実際に、$[0,1)$の一様分布から1000個のサンプルを取り出し、$95\%$の信頼度で、母集団平均を推計する。
図~\ref{populationMean}の横棒は、各サンプルで推定した母集団平均の範囲を表す。
縦の破線は、正しい母集団平均である。
\end{document}